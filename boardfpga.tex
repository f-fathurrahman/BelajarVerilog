\section{Eksperimen dengan hardware (FPGA)}

\subsection{Pengenalan IO}

{\color{red} Perlu gambar board FPGA yang akan digunakan}

Beberapa daftar port dari board FPGA yang digunakan:

\begin{table}[H]
\caption{Beberapa port yang digunakan pada board FPGA}\label{tab:pin}
\centering
\begin{tabular}{|c|c|}
\hline 
Clock & PIN\_23\\
\hline 
Reset Push Button & PIN\_25\\
\hline 
Buzzer & PIN\_110 \\
\hline
IR & PIN\_100 \\
\hline 
\multicolumn{2}{|c|}{Push Button}\\
\hline 
key1 & PIN\_88 \\
\hline 
key2 & PIN\_89 \\
\hline 
key3 & PIN\_90 \\
\hline 
key4 & PIN\_91 \\
\hline 
\multicolumn{2}{|c|}{LED} \\
\hline 
LED1 & PIN\_87 \\
\hline 
LED2 & PIN\_86 \\
\hline 
LED3 & PIN\_85 \\
\hline 
LED4 & PIN\_84 \\
\hline 
\end{tabular}
\par
\end{table}



Untuk PIN assignment, buat skematik atau file Verilog terlebih dahulu, kemudian compile.
Buka PIN assignment, set pin yang diperlukan.


\subsubsection{LED}

Assignment nilai boolean ke LED

\subsubsection{Push buttons}

\subsubsection{Seven segments}

\subsubsection{Buzzer}


\subsubsection{Clock}
{\color{red} Berapa frekuensi clock yang digunakan ? 50 MHz ?}

LED blinking (sudah menggunakan counter, implementasinya
mudah pada Verilog)

\subsubsection{IR}
Gunakan kode Verilog yang sudah ada.


\subsection{Rangkaian kombinasional}

XOR

half adder dan full adder

Menyalakan satu atau beberapa LED dengan kombinasi input
4 push button
\begin{itemize}
\item LED1 menyala jika button1 dan button3 ditekan atau button1 dan button 4 ditekan
\item LED2 menyala jika button2 dan button4 ditekan atau button1 ditekan
\end{itemize}





\subsection{Rangkaian sekuensial}

Implementasi D flip-flop

Implementasi J-K flip-flop dan
Toggle operation: buzzer dan LED

register, counter

Kalkulator sederhana, input dari remote IR ?

Rangkaian multiplexing LED

Menampilkan LED perdigit

IR
