\section{Simulasi rangkaian digital dengan Verilog}

Selain menggunakan Quartus Prime, kita juga dapat menggunakan tools lain untuk
mempelajari Verilog. Beberapa tools yang sering digunakan
adalah \textbf{Icarus Verilog} dan \textbf{GTKWave}. Icarus
Verilog adalah compiler Verilog gratis yang dapat digunakan
untuk simulasi rangkaian digital dengan Verilog. Simulasi ini
biasanya menghasilkan output waveform yang dapat divisualisasi dengan
menggunakan GTKWave.

Pada Ubuntu, Icarus Verilog dan GTKWave dapat diinstal dengan menggunakan perintah
\begin{minted}{text}
sudo apt install iverilog gtkwave
\end{minted}
Icarus Verilog dan GTKWave juga dapat digunakan pada sistem operasi Windows.

Kode Verilog yang ditulis untuk simulasi suatu rangkaian digital sering
disebut dengan \textit{textbench}. Modul yang diuji pada simulasi tersebut
disebut sebagai \textit{unit under testing}.

Sebagai contoh pertama, kita akan membuat kode Verilog untuk simulasi
modul {\tt\textbf{mux2}}. Buat file baru dengan nama {\tt test\_mux2.v}
yang isinya sebagai berikut.
\begin{mdframed}[backgroundcolor=mintedbg]
{\setstretch{1.0}
\inputminted[breaklines,fontsize=\small]{verilog}{codes/test_mux2.v}
}\end{mdframed}

Compile file ini dengan Icarus Verilog:
\begin{minted}[fontsize=\small]{text}
iverilog test_mux2.v -o test_mux2
\end{minted}

Perintah ini akan menghasilkan file baru dengan nama {\tt\small test\_mux2}.
File ini merupakan file teks yang dapat dieksekusi dengan perintah {\tt vvp}.
Pada sistem Linux, file ini sudah bersifat \textit{executable} dan dapat dieksekusi
langsung di terminal
\begin{minted}[fontsize=\small]{text}
> ./test_mux2
VCD info: dumpfile test_mux2.vcd opened for output.
time =     0, i1 = 0, i2 = 1, s = 1, o = 1
time =     1, i1 = 1, i2 = 0, s = 1, o = 0
time =     2, i1 = 1, i2 = 1, s = 1, o = 1
time =     3, i1 = 0, i2 = 1, s = 0, o = 0
time =     4, i1 = 1, i2 = 1, s = 0, o = 1
\end{minted}


Berikut ini adalah beberapa penjelasan kode di atas.

\begin{verilogcode}
`timescale 1ns / 1ps
\end{verilogcode}
Baris ini menyatakan bahwa satuan waktu dalam simulasi adalah 1 ns dan
resolusinya adalah 1 ps (0.001 ns).

\begin{verilogcode}
`include "mux2.v"
\end{verilogcode}
Baris ini hanya menyisipkan file {\tt mux2.v} yang berisi definisi dari modul {\tt mux2}.
Baris ini tidak diperlukan apabila \textit{testbench} diberikan pada file yang sama.
Beberapa tools simulasi dapat mencari definisi modul yang diperlukan sehingga
baris ini biasanya juga tidak diperlukan pada kasus tersebut.

\begin{verilogcode}
module test_mux2;
  reg i1, i2, s;
  wire o;
\end{verilogcode}

Baris pertama mendefinisikan suatu modul bernama {\tt\small test\_mux2}. Kode
ini tidak memiliki definisi port karena hanya dimaksudkan sebagai {\it testbench}
untuk modul {\tt\textbf{mux2}}. Pada baris selanjutnya,
tiga sinyal dengan nama {\tt\textbf{i1}}, {\tt\textbf{i2}},
dan {\tt\textbf{s}} dideklarasikan. Tiga sinyal ini nantinya akan menjadi input untuk
modul {\tt\textbf{mux2}}
yang nilainya dapat kita atur nanti.
Sebagai output dari modul {\tt\textbf{mux}}, satu sinyal dengan tipe {\tt\textbf{wire}}
didefinisikan pada baris selanjutnya.

\begin{verilogcode}
  mux2 uut( .x1(i1), .x2(i2), .s(s), .f(o) );
\end{verilogcode}

Pada baris ini, satu \textit{instance} dari modul {\tt\textbf{mux2}} didefinisikan
dengan menggunakan nama $uut$ (\textit{unit under test}).

Kode yang dieksekusi pada awal simulasi diberikan di dalam blok
{\tt\textbf{initial}}.

\begin{verilogcode}
  $dumpfile("test_mux2.vcd");
  $dumpvars(0, test_mux2);
\end{verilogcode}

Dua baris di atas menyatakan bahwa hasil simulasi akan akan ditulis
ke dalam \textit{waveform} yang disimpan pada file \textit{textit}.

\begin{verilogcode}
i1 = 1'b0;
i2 = 1'b1;
s  = 1'b1;
\end{verilogcode}

Karena potongan kode di atas diberikan di dalam blok {\tt\textbf{initial}}
maka kode di atas memberikan nilai input i1, i2, dan s pada saat awal
simulasi

{\color{red} TODO: Contoh simulasi dengan ModelSim}
