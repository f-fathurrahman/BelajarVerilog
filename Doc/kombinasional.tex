\subsection{Rangkaian kombinasional}

\subsubsection{Rangkaian XOR}

\textbf{TUGAS}

Buatlah rangkaian digital dengan spesifikasi sebagai berikut.
\begin{itemize}
\item Input dua buah push button, output 1 LED
\item LED akan menyala jika dan hanya jika 1 push button ditekan.
\end{itemize}

Implementasikan dengan menggukan gerbang logika AND, OR, dan NOT
(jika diperlukan). (Verilog + skematik)

\subsubsection{Pendeteksi ganjil}

Membuat rangkaian logika digital sederhana dengan spesifikasi sebagai berikut.

\begin{itemize}
\item input push button sebanyak 3 buah, misalkan bernama {\tt A}, {\tt B}
dan {\tt C}
\item output berupa satu LED dengan nama {\tt led}
\item LED akan menyala jika jumlah push button
yang ditekan ganjil
\end{itemize}

Misalkan ketika push button ditekan nilai logikanya adalah 1 dan 
LED akan menyala jika nilai logika adalah 1, maka dapat diperoleh tabel
sebagai berikut.

\begin{table}
\centering
\begin{tabular}{|c|c|c|c|c|}
\hline
 A  &  B  &  C  & LED \\
 \hline\hline
 0  &  0  &  0  &  0  \\
 \hline
 0  &  0  &  1  &  1  \\
 \hline
 0  &  1  &  0  &  1  \\
 \hline
 0  &  1  &  1  &  0  \\
 \hline
 1  &  0  &  0  &  1  \\
 \hline
 1  &  0  &  1  &  0  \\
 \hline
 1  &  1  &  0  &  0  \\
 \hline
 1  &  1  &  1  &  1  \\
 \hline
\end{tabular}
\par
\end{table}


\begin{equation*}
\mathrm{LED} = ABC + \bar{A}\bar{B}C + \bar{A}B\bar{C} + A\bar{B}\bar{C}
\end{equation*}

Kode Verilog yang mengimplementasikan persamaan logika di atas adalah
sebagai berikut.
\begin{verilogcode}
module deteksi_ganjil_3(
  input A, input B, input C,
  output led );

  assign led = ( A & B &  C) | (~A & ~B &  C) | 
               (~A & B & ~C) | ( A & ~B & ~C);

endmodule
\end{verilogcode}

\textbf{TUGAS} Buatlah desain (skematik atau Verilog) untuk mendeteksi apakah
jumlah push button yang ditekan adalah ganjil untuk \textbf{empat} buah
push button. Sederhanakan persamaan logika yang diperoleh jika diperlukan
dengan metode Karnaugh map.


\subsubsection{\textit{Converter} BCD ke \textit{seven-segment}}

Buatlah rangkaian digital untuk mengubah input BCD ke
output seven segment. Buat tabel kebenaran dan rangkaian (dalam skematik
atau Verilog struktural). Minimisasi persamaan logika yang diperoleh dengan menggunakan
metode Karnaugh map.

\begin{table}[H]
{\centering
\begin{tabular}{|c|c|c|c||c|c|c|c|c|c|c|c|}
\hline
\multicolumn{4}{|c||}{Input} & \multicolumn{8}{|c|}{Output} \\
\hline
PB1 & PB2 & PB3 & PB4 & a & b & c & d & e & f & g & dp \\
\hline
0 & 0 & 0 & 0 &  &  &  &  &  &  &  & \\
0 & 0 & 0 & 1 &  &  &  &  &  &  &  & \\
0 & 0 & 1 & 0 &  &  &  &  &  &  &  & \\
... & ... & ... & ... &  &  &  &  &  &  &  & \\
1 & 1 & 1 & 1 &  &  &  &  &  &  &  & \\
\hline
\end{tabular}
\par}
\end{table}

