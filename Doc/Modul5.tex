\documentclass[a4paper,12pt,bahasa]{extarticle}
\usepackage[a4paper]{geometry}
\geometry{verbose,tmargin=2cm,bmargin=2cm,lmargin=2cm,rmargin=2cm}

\usepackage{fontspec}
\setmainfont{Calibri}
\setmonofont{Courier 10 Pitch}

\setlength{\parindent}{0cm}

\usepackage{amsmath}

\usepackage{setspace}
\onehalfspacing

\usepackage{xcolor}
\usepackage{float}
\usepackage{graphicx}
\usepackage{hyperref}
\usepackage{url}

\usepackage{float}

\usepackage{minted}
%\newminted{verilog}{breaklines}
\newminted{verilog}{breaklines,fontsize=\small}
\newminted{text}{breaklines,fontsize=\small}

\usepackage{babel}

\definecolor{mintedbg}{rgb}{0.95,0.95,0.95}
\usepackage{mdframed}

%\global\mdfapptodefinestyle{verilognotes}{%
%rightline=true,%
%innerleftmargin=10,%
%innerrightmargin=10,%
%frametitlerule=true,%
%frametitlerulewidth=2pt}

\mdfdefinestyle{verilognotes}{%
  linecolor=blue,
  linewidth=1pt,
  frametitlerule=true,
  frametitlerulewidth=2pt
}

\BeforeBeginEnvironment{minted}{\begin{mdframed}[backgroundcolor=mintedbg]}
\AfterEndEnvironment{minted}{\end{mdframed}}


\begin{document}


\title{MODUL V \\
Implementasi Rangkaian Digital dengan FPGA}
\author{}
\date{}
\maketitle

\section{Tujuan}

Tujuan dari praktikum ini adalah sebagai berikut,
\begin{enumerate}
\item mengenal teknik pemrograman FPGA dengan menggunakan bahasa deskripsi hardware (Verilog)
dan skematik
\item mengenal gerbang-gerbang logika dan membuat rangkaian digital sederhana dengan
skematik dan bahasa deskripsi hardware
\end{enumerate}



\section{Alat dan Bahan}

\begin{enumerate}
\item Komputer / laptop dengan aplikasi Quartus Prime
\item Board FPGA (Altera Cyclone IV E)
\end{enumerate}

\section{Dasar Teori}

programmable logic devices dan FPGA

Entri desain

\subsection{Pengenalan Quartus Prime}

Quartus Prime merupakan perangkat lunak CAD (computer-assisted design)
yang digunakan untuk desain rangkaian digital. Quartus Prime dikembangkan
oleh Altera. Versi Lite dari Quartus Prime dapat diunduh secara gratis
pada laman Altera\footnote{\url{http://dl.altera.com/?edition=lite}}.
Quartus Prime dapat dijalankan pada
platform Windows dan Linux.
Jendela utama dari Quartus Prime Lite dapat dilihat pada Gambar
\ref{fig:main_window}.

\begin{figure}
\centering
\includegraphics[width=\textwidth]{images/FirstOpen.png}
\par
\caption{Tampilan jendela utama Quartus Prime Lite}\label{fig:main_window}
\end{figure}

Dengan menggunakan CAD, setidaknya ada dua cara untuk
mendesain rangkaian digital:
\begin{itemize}
\item \textit{schematic capture}, dengan membuat skematik dari rangkaian yang
diinginkan.
\item menggunakan Hardware Description Language (HDL).
Dua jenis HDL yang paling populer adalah Verilog dan VHDL.
Kedua bahasa tersebut telah diadopsi sebagai IEEE Standard.
Pada tulisan ini akan digunakan Verilog.
\end{itemize}

Kita akan mulai dengan menggunakan skematik.

\subsection{Menambahkan skematik baru}

Skematik baru dapat ditambahkan ke dalam project dengan memilih menu
\textbf{File $\rightarrow$ New}. Pilih \textbf{New Diagram/Schematic File}, kemudian
\textbf{OK}.

\begin{figure}[H]
\centering
\includegraphics[scale=0.5]{images/NewSchematic.png}
\par
\end{figure}

File skematik kosong akan terbuka pada tab baru dengan nama \textbf{Block1.bdf}.
Kita dapat membuat skematik yang kita inginkan pada file ini.

\begin{figure}[H]
\centering
\includegraphics[scale=0.5]{images/EmptySchematic.png}
\par
\end{figure}

Untuk menambahkan komponen, dapat dilakukan dengan cara mengklik toolbar
\textbf{Symbol Tool}.

\begin{figure}[H]
\centering
\includegraphics[scale=0.5]{images/SymbolTool.png}
\par
\end{figure}

Komponen yang ingin ditambahkan pada skematik dapat diperoleh dengan
ekspasi node \textbf{Libraries}, mencari komponen tersebut, dan memilihnya.
Misalkan kita ingin menambahkan gerbang AND dengan dua input, maka dapat dipilih
pada \textbf{primitives $\rightarrow$ logic $\rightarrow$ and2}. Klik
\textbf{OK} setelah komponen yang diinginkan telah dipilih.

\begin{figure}[H]
\centering
\includegraphics[scale=0.5]{images/BlockAnd.png}
\par
\end{figure}

Pemilihan komponen juga dapat dilakukan dengan mengetikkan nama komponen yang
diinginakan pada isian \textbf{Name}, misalnya \textbf{jkff} untuk J-K flip-flop.

Khusus untuk menambahkan komponen input dan output, dapat juga digunakan
toolbar \textbf{Pin Tool}.
\begin{figure}[H]
\centering
\includegraphics[scale=0.5]{images/PinTool.png}
\par
\end{figure}

Untuk menghubungkan antara satu komponen dengan komponen yang lain, dapat
digunakan \textbf{Orthogonal Node Tool}.
\begin{figure}[H]
\centering
\includegraphics[scale=0.5]{images/OrthogonalNodeTool.png}
\par
\end{figure}

Berikut ini adalah contoh skematik untuk multiplexer 2-to-1:
\begin{figure}[H]
\centering
\includegraphics[width=\textwidth]{images/sch_mux_2_1.png}
\par
\end{figure}

Skematik ini kemudian dapat digunakan untuk proses lebih lanjut seperti
simulasi dan download ke hardware FPGA.




\input{verilog1}

\section{Praktikum}

Berikut ini adalah board FPGA yang akan kita gunakan dalam praktikum ini.

\begin{figure}[H]
{\centering
\includegraphics[width=0.5\textwidth]{images/foto_FPGA_v1.jpg}
\par}
\caption{FPGA Board Versi A}
\end{figure}

{\color{red}Perlu gambar FPGA yang kedua dan informasi PIN untuk board kedua.}

Berikut ini adalah pedoman umum yang akan kita lakukan pada praktikum ini.

\begin{itemize}

\item Buat proyek Quartus Prime baru, misalkan dengan nama \textbf{Modul5\_Kelompok\_XX}.
di dalam direktori dengan nama yang sama. \textit{Hindari penggunaan
spasi dalam nama proyek}.

\item Langkah untuk menambahkan file skematik baru dapat dilakukan sesuai dengan langkah yang
dijelaskan pada bagian \ref{subsec:skematik}. Penambahan file Verilog
juga dapat dilakukan dengan cara yang mirip.

\item Sebelum melakukan kompilasi desain (skematik atau Verilog)
pastikan bahwa desain tersebut telah dibuat menjadi top-level entity
\footnote{Top-level entity memiliki peran yang sama
dengan fungsi \texttt{main} pada bahasa C/C++}
melalui menu \textbf{Project $\rightarrow$ Set as Top-Level Entity} atau
keyboard shortcut \textbf{Ctrl+Shift+J}.

\item Lakukan kompilasi setelah desain selesai dibuat.

\item Lakukan PIN assignment. Pastikan bahwa semua port yang akan digunakan
telah diberikan PIN yang sesuai.

\item Lakukan kompilasi sekali lagi.

\item Download desain yang telah dibuat ke FPGA dan verifikasi apakah telah berjalan
sesuai dengan yang diharapkan.
\end{itemize}

Langkah-langkah tersebut akan diilustrasikan pada eksperimen dengan LED.

\subsection{Pengenalan IO: LED, push button dan seven segment}

\subsubsection{LED}

LED merupakan output yang paling sederhana pada board FPGA yang akan
kita gunakan. Salah satu tujuan utama dari eksperimen ini adalah
untuk mengetahui keadaan LED apabila diberikan nilai logika 0 dan 1.
Hal ini lebih mudah dilakukan dengan menggunakan Verilog.
Kita juga akan mempelajari sintaks Verilog untuk


Buat file baru, misalnya dengan nama \textbf{led\_light.v}.
\textit{Hindari nama file dengan menggunakan spasi}.
Kemudian ketik kode Verilog berikut pada file tersebut.
Kode Verilog ini memberikan nilai logika ke tiap
LED (hardwired, tanpa ada input).

{\setstretch{1.0}
\begin{verilogcode}
module led_light(led);
  output [3:0] led;
  assign led = 4'b0000; // coba ubah-ubah nilai ini untuk tiap LED
endmodule
\end{verilogcode}
}

Compile file ini dengan cara klik icon \textbf{Compile} atau dengan menu
\textbf{Processing $\rightarrow$ Start Compilation} atau menggunakan shortcut
\textbf{Ctrl + L}.

Jika tidak ada kesalahan pada saat proses kompilasi,
maka langkah selanjutnya adalah melakukan
PIN assignment, yang dapat dilakukan dengan memilih menu
\textbf{Assignment $\rightarrow$ Pin Planner} atau menggunakan shortcut
\textbf{Ctrl + N}.

Berikut ini adalah data PIN yang diberikan.

\begin{table}[H]
\centering
\begin{tabular}{|c|c|}
\hline 
\multicolumn{2}{|c|}{LED} \\
\hline 
LED1 & PIN\_87 \\
\hline 
LED2 & PIN\_86 \\
\hline 
LED3 & PIN\_85 \\
\hline 
LED4 & PIN\_84 \\
\hline 
\end{tabular}
\par
\end{table}

Atur PIN assigment sesuai dengan keperluan.
Jendela PIN assignment ini dapat dibiarkan terbuka.

\begin{figure}[H]
\centering
\includegraphics[scale=0.5]{images/PinPlanner_4LED.png}
\par
\caption{Contoh PIN Assignment untuk 4 LED, mohon disesuaikan dengan
kode Verilog dan board yang digunakan}
\end{figure}

\textit{Compile lagi file tersebut setelah PIN assignment dilakukan}.

Jika tidak ada pesan error, langkah selanjutnya adalah mendownload program
ini ke FPGA. Proses ini dapat dilakukan dengan cara memilih menu
\textbf{Tools $\rightarrow$ Programmer}.
Klik button \textbf{Add File} untuk menambahkan file {\tt led\_light.sof}.
File ini biasanya ada di dalam subdirektori {\tt output} dari direktori
proyek.
Pastikan juga hardware telah terdeteksi. Jika belum terdeteksi, tambahkan melalui
dengan mengklik button \textbf{Hardware Setup}.

\begin{figure}[H]
\centering
\includegraphics[scale=0.4]{images/Programmer_4LED.png}
\par
\caption{Tampilan tool \textbf{Programmer}}
\end{figure}

Misalkan pada board FPGA yang digunakan urutan LED dari kiri ke kanan adalah LED1,
LED2, LED3, dan LED4.

Misalkan memberikan assignment sebagai berikut.
\begin{itemize}
\item LED1 diwakili dengan {\tt led[0]}
\item LED2 diwakili dengan {\tt led[1]}
\item LED3 diwakili dengan {\tt led[2]}
\item LED4 diwakili dengan {\tt led[3]}
\end{itemize}


Misalkan juga kita memberikan nilai logika pada {\tt led} dengan
kode Verilog berikut.
\begin{verilogcode}
  led = 4'b1010;
\end{verilogcode}
Maka nilai 0 (nilai bit paling kanan atau LSB) diberikan pada {\tt led[0]}
atau LED1. Nilai pada bit kedua dari kanan diberikan untuk {\tt led[1]}, bit ketiga
untuk {\tt led[2]}, dan bit keempat (paling kiri atau MSB) untuk {\tt led[3]}.

Bagian {\tt output [3:0] led} pada kode di atas dapat diganti
dengan {\tt output [1:4] led} untuk memudahkan assignment nilai logika
sesuai dengan urutan LED di board yang digunakan. Sehingga kita dapat
melakukan assigment sebagai berikut.
\begin{itemize}
\item LED1 diwakili dengan {\tt led[1]}
\item LED2 diwakili dengan {\tt led[2]}
\item LED3 diwakili dengan {\tt led[3]}
\item LED4 diwakili dengan {\tt led[4]}
\end{itemize}


Cobalah bereksprimen dengan cara mengganti-ganti nilai logika
dari {\tt led}, kemudian isilah tabel berikut.

\begin{table}[H]
\centering
\begin{tabular}{|c|c|}
\hline
Nilai logika & Keadaan LED (on/off) \\
\hline
0 & \\
1 & \\
\hline
\end{tabular}
\par
\end{table}

Untuk penggunaan berikutnya, isikan juga PIN untuk LED, sesuai dengan
letaknya di board FPGA. Dari kiri ke kanan:

\begin{table}[H]
{\centering
\begin{tabular}{|c|c|c|c|}
\hline
PIN: \hspace{2cm} & PIN: \hspace{2cm} & PIN: \hspace{2cm} & PIN: \hspace{2cm} \\
\hline
\end{tabular}
\par}
\end{table}

\subsubsection{Push buttons}

Setelah mengetahui keadaan LED jika diberikan nilai logika 0 dan 1, kita
dapat mengetahui keadaan logika push button apabila dilepas dan ditekan.

Buat file Verilog baru dengan mendefiniskan satu modul
dengan input dari push button dan output ke LED.

\begin{verilogcode}
module test_buttons( buttons, led );
  input  [3:0] buttons;
  output [3:0] led;

  assign led = buttons;
endmodule
\end{verilogcode}

Bisa juga menggunakan potongan kode berikut.
\begin{verilogcode}
  assign led[0] = buttons[0];
  assign led[1] = buttons[1];
  assign led[2] = buttons[2];
  assign led[3] = buttons[3];
\end{verilogcode}

Cobalah bereksperimen dengan kode Verilog yang ada dan juga menggukan operator Verilog
seperti 

\begin{table}[H]
\centering
\begin{tabular}{|c|c|}
\hline
Nilai logika & Keadaan PB \\
\hline
0 & \\
1 & \\
\hline
\end{tabular}
\par
\end{table}

Untuk penggunaan berikutnya, isikan juga PIN untuk push button,
sesuai dengan letaknya di board FPGA. Dari kiri ke kanan:

\begin{table}[H]
{\centering
\begin{tabular}{|c|c|c|c|}
\hline
PIN: \hspace{2cm} & PIN: \hspace{2cm} & PIN: \hspace{2cm} & PIN: \hspace{2cm} \\
\hline
\end{tabular}
\par}
\end{table}


\subsubsection{Seven segments}

Lihat juga Modul 1.

Berikut ini adalah tabel nomor PIN terkait dengan seven-segment display
yang diberikan oleh distributor FPGA.

\begin{table}[H]
\centering
\begin{tabular}{|c|c|}
\hline 
\multicolumn{2}{|c|}{Seven-segment LED} \\
\hline 
DIG1 & PIN\_133 \\
\hline 
DIG2 & PIN\_135 \\
\hline 
DIG3 & PIN\_136 \\
\hline 
DIG4 & PIN\_137 \\
\hline 
SEG0 & PIN\_128 \\
\hline 
SEG1 & PIN\_121 \\
\hline 
SEG2 & PIN\_125 \\
\hline
SEG3 & PIN\_129 \\
\hline
SEG4 & PIN\_132 \\
\hline
SEG5 & PIN\_126 \\
\hline
SEG6 & PIN\_124 \\
\hline
SEG7 & PIN\_127 \\
\hline 
\end{tabular}
\par
\end{table}

Seperti pada percobaan sebelumnya, buat file Verilog dengan konten sebagai berikut.
Lengkapi kode tersebut dan modifikasi bagian yang Anda perlukan.

\begin{verilogcode}
module tes_sseg(
  output [7:0] sseg,
  output [3:0] en_dig
);

  assign sseg[7] = 0;
  // lengkapi jika diperlukan
  assign sseg[0] = 1;
  
  assign en_dig[3] = 1;
endmodule
\end{verilogcode}

\begin{figure}[H]
\centering
\includegraphics[scale=0.5]{images/sseg.png}
\par
\caption{Seven segment}\label{fig:sseg}
\end{figure}

Untuk referensi lebih lanjut isilah tabel berikut dengan memperhatikan Gambar
\ref{fig:sseg}.

\begin{table}[H]
\centering
\begin{tabular}{|c|c|}
\hline
Segmen & PIN \\
\hline
A & \\
\hline
B & \\
\hline
C & \\
\hline
D & \\
\hline
E & \\
\hline
F & \\
\hline
G & \\
\hline
DP & \\
\hline
\end{tabular}
\end{table}


\subsection{Rangkaian kombinasional}

\subsubsection{Rangkaian XOR}

\textbf{TUGAS}

Buatlah rangkaian digital dengan spesifikasi sebagai berikut.
\begin{itemize}
\item Input dua buah push button, output 1 LED
\item LED akan menyala jika dan hanya jika 1 push button ditekan.
\end{itemize}

Implementasikan dengan menggukan gerbang logika AND, OR, dan NOT
(jika diperlukan). (Verilog + skematik)

\subsubsection{Pendeteksi ganjil}

Membuat rangkaian logika digital sederhana dengan spesifikasi sebagai berikut.

\begin{itemize}
\item input push button sebanyak 3 buah, misalkan bernama {\tt A}, {\tt B}
dan {\tt C}
\item output berupa satu LED dengan nama {\tt led}
\item LED akan menyala jika jumlah push button
yang ditekan ganjil
\end{itemize}

Misalkan ketika push button ditekan nilai logikanya adalah 1 dan 
LED akan menyala jika nilai logika adalah 1, maka dapat diperoleh tabel
sebagai berikut.

\begin{table}
\centering
\begin{tabular}{|c|c|c|c|c|}
\hline
 A  &  B  &  C  & LED \\
 \hline\hline
 0  &  0  &  0  &  0  \\
 \hline
 0  &  0  &  1  &  1  \\
 \hline
 0  &  1  &  0  &  1  \\
 \hline
 0  &  1  &  1  &  0  \\
 \hline
 1  &  0  &  0  &  1  \\
 \hline
 1  &  0  &  1  &  0  \\
 \hline
 1  &  1  &  0  &  0  \\
 \hline
 1  &  1  &  1  &  1  \\
 \hline
\end{tabular}
\par
\end{table}


\begin{equation*}
\mathrm{LED} = ABC + \bar{A}\bar{B}C + \bar{A}B\bar{C} + A\bar{B}\bar{C}
\end{equation*}

Kode Verilog yang mengimplementasikan persamaan logika di atas adalah
sebagai berikut.
\begin{verilogcode}
module deteksi_ganjil_3(
  input A, input B, input C,
  output led );

  assign led = ( A & B &  C) | (~A & ~B &  C) | 
               (~A & B & ~C) | ( A & ~B & ~C);

endmodule
\end{verilogcode}

\textbf{TUGAS} Buatlah desain (skematik atau Verilog) untuk mendeteksi apakah
jumlah push button yang ditekan adalah ganjil untuk \textbf{empat} buah
push button. Sederhanakan persamaan logika yang diperoleh jika diperlukan
dengan metode Karnaugh map.


\subsubsection{\textit{Converter} BCD ke \textit{seven-segment}}

Buatlah rangkaian digital untuk mengubah input BCD ke
output seven segment. Buat tabel kebenaran dan rangkaian (dalam skematik
atau Verilog struktural). Minimisasi persamaan logika yang diperoleh dengan menggunakan
metode Karnaugh map.

\begin{table}[H]
{\centering
\begin{tabular}{|c|c|c|c||c|c|c|c|c|c|c|c|}
\hline
\multicolumn{4}{|c||}{Input} & \multicolumn{8}{|c|}{Output} \\
\hline
PB1 & PB2 & PB3 & PB4 & a & b & c & d & e & f & g & dp \\
\hline
0 & 0 & 0 & 0 &  &  &  &  &  &  &  & \\
0 & 0 & 0 & 1 &  &  &  &  &  &  &  & \\
0 & 0 & 1 & 0 &  &  &  &  &  &  &  & \\
... & ... & ... & ... &  &  &  &  &  &  &  & \\
1 & 1 & 1 & 1 &  &  &  &  &  &  &  & \\
\hline
\end{tabular}
\par}
\end{table}




\section{Tugas Pendahuluan}

\begin{enumerate}

\item Carilah informasi umum mengenai FPGA:
\begin{itemize}
\item penggunaan FPGA
\item perbandingan FPGA dengan mikrokontroler, kelebihan dan kekurangan
\end{itemize}

\item Carilah informasi mengenai bahasa deksripsi hardware, terutama Verilog.

\item Lengkapi tabel kebenaran untuk konversi dari BCD ke seven segment.
\end{enumerate}


\end{document}

