\documentclass[a4paper,12pt,bahasa]{extarticle}
\usepackage[a4paper]{geometry}
\geometry{verbose,tmargin=2cm,bmargin=2cm,lmargin=2cm,rmargin=2cm}

\usepackage{fontspec}
\setmainfont{Calibri}
\setmonofont{Courier 10 Pitch}

\setlength{\parindent}{0cm}

\usepackage{amsmath}

\usepackage{setspace}
\onehalfspacing

\usepackage{xcolor}
\usepackage{float}
\usepackage{graphicx}
\usepackage{hyperref}
\usepackage{url}

\usepackage{float}

\usepackage{minted}
%\newminted{verilog}{breaklines}
\newminted{verilog}{breaklines,fontsize=\small}
\newminted{text}{breaklines,fontsize=\small}

\usepackage{babel}

\definecolor{mintedbg}{rgb}{0.95,0.95,0.95}
\usepackage{mdframed}

%\global\mdfapptodefinestyle{verilognotes}{%
%rightline=true,%
%innerleftmargin=10,%
%innerrightmargin=10,%
%frametitlerule=true,%
%frametitlerulewidth=2pt}

\mdfdefinestyle{verilognotes}{%
  linecolor=blue,
  linewidth=1pt,
  frametitlerule=true,
  frametitlerulewidth=2pt
}

%\BeforeBeginEnvironment{minted}{\begin{mdframed}[backgroundcolor=mintedbg]}
%\AfterEndEnvironment{minted}{\end{mdframed}}


\begin{document}


\title{MODUL V \\
Implementasi Rangkaian Digital dengan FPGA}
\author{}
\date{}
\maketitle

\section{Tujuan}

\section{Alat dan Bahan}


\section{Dasar Teori}

programmable logic devices dan FPGA

Entri desain

\subsection{Pengenalan Quartus Prime}

Quartus Prime merupakan perangkat lunak CAD (computer-assisted design)
yang digunakan untuk desain rangkaian digital. Quartus Prime dikembangkan
oleh Altera. Versi Lite dari Quartus Prime dapat diunduh secara gratis
pada laman Altera\footnote{\url{http://dl.altera.com/?edition=lite}}.
Quartus Prime dapat dijalankan pada
platform Windows dan Linux.
Jendela utama dari Quartus Prime Lite dapat dilihat pada Gambar
\ref{fig:main_window}.

\begin{figure}
\centering
\includegraphics[width=\textwidth]{images/FirstOpen.png}
\par
\caption{Tampilan jendela utama Quartus Prime Lite}\label{fig:main_window}
\end{figure}

Dengan menggunakan CAD, setidaknya ada dua cara untuk
mendesain rangkaian digital:
\begin{itemize}
\item \textit{schematic capture}, dengan membuat skematik dari rangkaian yang
diinginkan.
\item menggunakan Hardware Description Language (HDL).
Dua jenis HDL yang paling populer adalah Verilog dan VHDL.
Kedua bahasa tersebut telah diadopsi sebagai IEEE Standard.
Pada tulisan ini akan digunakan Verilog.
\end{itemize}

Kita akan mulai dengan menggunakan skematik.

\subsection{Menambahkan skematik baru}

Skematik baru dapat ditambahkan ke dalam project dengan memilih menu
\textbf{File $\rightarrow$ New}. Pilih \textbf{New Diagram/Schematic File}, kemudian
\textbf{OK}.

\begin{figure}[H]
\centering
\includegraphics[scale=0.5]{images/NewSchematic.png}
\par
\end{figure}

File skematik kosong akan terbuka pada tab baru dengan nama \textbf{Block1.bdf}.
Kita dapat membuat skematik yang kita inginkan pada file ini.

\begin{figure}[H]
\centering
\includegraphics[scale=0.5]{images/EmptySchematic.png}
\par
\end{figure}

Untuk menambahkan komponen, dapat dilakukan dengan cara mengklik toolbar
\textbf{Symbol Tool}.

\begin{figure}[H]
\centering
\includegraphics[scale=0.5]{images/SymbolTool.png}
\par
\end{figure}

Komponen yang ingin ditambahkan pada skematik dapat diperoleh dengan
ekspasi node \textbf{Libraries}, mencari komponen tersebut, dan memilihnya.
Misalkan kita ingin menambahkan gerbang AND dengan dua input, maka dapat dipilih
pada \textbf{primitives $\rightarrow$ logic $\rightarrow$ and2}. Klik
\textbf{OK} setelah komponen yang diinginkan telah dipilih.

\begin{figure}[H]
\centering
\includegraphics[scale=0.5]{images/BlockAnd.png}
\par
\end{figure}

Pemilihan komponen juga dapat dilakukan dengan mengetikkan nama komponen yang
diinginakan pada isian \textbf{Name}, misalnya \textbf{jkff} untuk J-K flip-flop.

Khusus untuk menambahkan komponen input dan output, dapat juga digunakan
toolbar \textbf{Pin Tool}.
\begin{figure}[H]
\centering
\includegraphics[scale=0.5]{images/PinTool.png}
\par
\end{figure}

Untuk menghubungkan antara satu komponen dengan komponen yang lain, dapat
digunakan \textbf{Orthogonal Node Tool}.
\begin{figure}[H]
\centering
\includegraphics[scale=0.5]{images/OrthogonalNodeTool.png}
\par
\end{figure}

Berikut ini adalah contoh skematik untuk multiplexer 2-to-1:
\begin{figure}[H]
\centering
\includegraphics[width=\textwidth]{images/sch_mux_2_1.png}
\par
\end{figure}

Skematik ini kemudian dapat digunakan untuk proses lebih lanjut seperti
simulasi dan download ke hardware FPGA.




\input{verilog1}

\section{Praktikum}

\subsection{Pengenalan IO: LED dan push button}

{\color{red} Perlu gambar board FPGA yang akan digunakan}

Beberapa daftar port dari board FPGA yang digunakan:

\begin{table}[H]
\caption{Beberapa port yang digunakan pada board FPGA}\label{tab:pin}
\centering
\begin{tabular}{|c|c|}
\hline 
Clock & PIN\_23\\
\hline 
Reset Push Button & PIN\_25\\
\hline 
Buzzer & PIN\_110 \\
\hline
IR & PIN\_100 \\
\hline 
\multicolumn{2}{|c|}{Push Button}\\
\hline 
key1 & PIN\_88 \\
\hline 
key2 & PIN\_89 \\
\hline 
key3 & PIN\_90 \\
\hline 
key4 & PIN\_91 \\
\hline 
\multicolumn{2}{|c|}{LED} \\
\hline 
LED1 & PIN\_87 \\
\hline 
LED2 & PIN\_86 \\
\hline 
LED3 & PIN\_85 \\
\hline 
LED4 & PIN\_84 \\
\hline 
\end{tabular}
\par
\end{table}



Untuk PIN assignment, buat skematik atau file Verilog terlebih dahulu, kemudian compile.
Buka PIN assignment, set pin yang diperlukan.

Prosedur ini diberikan pada contoh menyalakan LED.

\subsubsection{LED}

Buat proyek baru dengan nama \textbf{led\_light}, misalnya.
Kemudian tambahkan file Verilog berikut
ke project. Kode Verilog ini memberikan nilai logika ke tiap
LED (hardwired, tanpa ada input).


{\setstretch{1.0}
\begin{verilogcode}
module led_light(led);
  output[3:0] led;
  assign led = 4'b0000; // coba ubah-ubah nilai ini untuk tiap LED
endmodule
\end{verilogcode}
}

Compile file ini dengan cara klik icon \textbf{Compile} atau dengan menu
\textbf{Processing -> Start Compilation} atau menggunakan shortcut
\textbf{Ctrl + L}.

Jika tidak ada kesalahan pada saat proses kompilasi,
maka langkah selanjutnya adalah melakukan
PIN assignment, yang dapat dilakukan dengan memilih menu
\textbf{Assignment -> Pin Planner} atau menggunakan shortcut
\textbf{Ctrl + N}. Atur PIN assigment sesuai dengan Tabel \ref{tab:pin}.

\begin{figure}[H]
\centering
\includegraphics[scale=0.5]{images/PinPlanner_4LED.png}
\par
\caption{PIN Assignment untuk 4 LED}
\end{figure}

Compile lagi file tersebut.

Jika tidak ada pesan error, langkah selanjutnya adalah mendownload program
ini ke FPGA. Proses ini dapat dilakukan dengan cara memilih menu
\textbf{Tools -> Programmer}.
Klik button \textbf{Add File} untuk menambahkan file {\tt led\_light.sof}.
File ini biasanya ada di dalam subdirektori {\tt output} dari direktori
proyek.
Pastikan juga hardware telah terdeteksi. Jika belum terdeteksi, tambahkan melalui
dengan mengklik button \textbf{Hardware Setup}.

\begin{figure}[H]
\centering
\includegraphics[scale=0.6]{images/Programmer_4LED.png}
\par
\caption{Tampilan tool \textbf{Programmer}}
\end{figure}

\begin{mdframed}[style=verilognotes,frametitle={Catatan Verilog}]
Misalkan pada board FPGA yang digunakan urutan LED dari kiri ke kanan adalah LED1,
LED2, LED3, dan LED4.

Misalkan memberikan assignment sebagai berikut.
\begin{itemize}
\item LED1 diwakili dengan {\tt led[0]}
\item LED2 diwakili dengan {\tt led[1]}
\item LED3 diwakili dengan {\tt led[2]}
\item LED4 diwakili dengan {\tt led[3]}
\end{itemize}


Misalkan juga kita memberikan nilai logika pada {\tt led} dengan
kode Verilog berikut.
\begin{verilogcode}
  led = 4'b1010;
\end{verilogcode}
Maka nilai 0 (nilai bit paling kanan atau LSB) diberikan pada {\tt led[0]}
atau LED1. Nilai pada bit kedua dari kanan diberikan untuk {\tt led[1]}, bit ketiga
untuk {\tt led[2]}, dan bit keempat (paling kiri atau MSB) untuk {\tt led[3]}.

Bagian {\tt output [3:0] led} pada kode di atas dapat diganti
dengan {\tt output [1:4] led} untuk memudahkan assignment nilai logika
sesuai dengan urutan LED di board yang digunakan. Sehingga kita dapat
melakukan assigment sebagai berikut.
\begin{itemize}
\item LED1 diwakili dengan {\tt led[1]}
\item LED2 diwakili dengan {\tt led[2]}
\item LED3 diwakili dengan {\tt led[3]}
\item LED4 diwakili dengan {\tt led[4]}
\end{itemize}

\end{mdframed}

Cobalah bereksprimen dengan cara mengganti-ganti nilai logika
dari {\tt led}, kemudian isilah tabel berikut.

\begin{table}[H]
\centering
\begin{tabular}{|c|c|}
\hline
Nilai logika & Keadaan LED (on/off) \\
\hline
0 & \\
1 & \\
\hline
\end{tabular}
\par
\end{table}


\subsubsection{Push buttons}

Buat project baru, dan buat file Verilog dengan mendefiniskan satu modul
dengan input dari push button dan output ke LED.
\begin{verilogcode}
module test_buttons( buttons, led );
  input  [3:0] buttons;
  output [3:0] led;

  assign led = buttons;
endmodule
\end{verilogcode}

Bisa juga menggunakan potongan kode berikut.
\begin{verilogcode}
  assign led[0] = buttons[0];
  assign led[1] = buttons[1];
  assign led[2] = buttons[2];
  assign led[3] = buttons[3];
\end{verilogcode}

Cobalah bereksperimen dengan kode Verilog yang ada dan juga menggukan operator Verilog
seperti 

\begin{table}[H]
\centering
\begin{tabular}{|c|c|}
\hline
Nilai logika & Keadaan PB \\
\hline
0 & \\
1 & \\
\hline
\end{tabular}
\par
\end{table}

\subsubsection{Seven segments}

Lihat Modul 1.



\subsection{Rangkaian kombinasional}

Membuat rangkaian logika digital sederhana dengan spesifikasi sebagai berikut.

\begin{itemize}
\item input push button sebanyak 3 buah, misalkan bernama {\tt A}, {\tt B}
dan {\tt C}
\item output berupa satu LED dengan nama {\tt led}
\item LED akan menyala jika jumlah push button
yang ditekan ganjil
\end{itemize}

Misalkan ketika push button ditekan nilai logikanya adalah 0 dan 
LED akan menyala jika nilai logika adalah 0, maka dapat diperoleh tabel
sebagai berikut.

\begin{table}
\centering
\begin{tabular}{|c|c|c|c|c|}
\hline
 A  &  B  &  C  & LED \\
 \hline\hline
 0  &  0  &  0  &  0  \\
 \hline
 0  &  0  &  1  &  1  \\
 \hline
 0  &  1  &  0  &  1  \\
 \hline
 0  &  1  &  1  &  0  \\
 \hline
 1  &  0  &  0  &  1  \\
 \hline
 1  &  0  &  1  &  0  \\
 \hline
 1  &  1  &  0  &  0  \\
 \hline
 1  &  1  &  1  &  1  \\
 \hline
\end{tabular}
\par
\end{table}


\begin{equation*}
LED = ABC + \bar{A}\bar{B}C + \bar{A}B\bar{C} + A\bar{B}\bar{C}
\end{equation*}

Kode Verilog yang mengimplementasikan persamaan logika di atas adalah
sebagai berikut.
\begin{verilogcode}
module deteksi_ganjil_3(
  input wire A,
  input wire B,
  input wire C,
  output wire led );

  assign led = (A & B & C) | (~A & ~B & C) | (~A & B & ~C) | (A & ~B & ~C);

endmodule
\end{verilogcode}


\subsection{BCD decoder}


\subsection{Rangkaian kombinasional}

- rangkaian pendeteksi genap ganjil, input 4 PB, output 1 LED

- Input PB -> BCD -> seven segment

- IR -> BCD -> seven segment

- Implementasi XOR

- half adder dan full adder

- Menyalakan satu atau beberapa LED dengan kombinasi input
4 push button yang diberikan.
\begin{itemize}
\item LED1 menyala jika button1 dan button3 ditekan atau button1 dan button 4 ditekan
\item LED2 menyala jika button2 dan button4 ditekan atau button1 ditekan
\end{itemize}

- Input BCD (dari ) ke output seven segment. Buat tabel kebenaran dan rangkaian (dalam skematik
atau Verilog struktural).

{\centering
\begin{tabular}{|c|c|c|c||c|c|c|c|c|c|c|c|}
\hline
\multicolumn{4}{|c||}{Input} & \multicolumn{8}{|c|}{Output} \\
\hline
PB1 & PB2 & PB3 & PB4 & a & b & c & d & e & f & g & dp \\
\hline
0 & 0 & 0 & 0 &  &  &  &  &  &  &  & \\
0 & 0 & 0 & 1 &  &  &  &  &  &  &  & \\
0 & 0 & 1 & 0 &  &  &  &  &  &  &  & \\
... & ... & ... & ... &  &  &  &  &  &  &  & \\
1 & 1 & 1 & 1 &  &  &  &  &  &  &  & \\
\hline
\end{tabular}
\par}


\subsection{Rangkaian sekuensial}

- Implementasi D flip-flop

- Implementasi J-K flip-flop dan

- Implementasi T flip-flip. Toggle operation: buzzer dan LED

- register, counter

- Kalkulator sederhana, input dari remote IR ?

- Rangkaian multiplexing LED

- Menampilkan LED perdigit


\section{Tugas Pendahuluan}


\end{document}
