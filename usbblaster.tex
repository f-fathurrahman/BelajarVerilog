\section{Persiapan USB-Blaster pada Linux}

Diperlukan untuk hardware.

Berikut ini adalah referensi yang dapat digunakan:
{\small
\begin{itemize}
\item \url{https://www.altera.com/support/support-resources/download/drivers/dri-usb_b-lnx.html}
\item \url{http://www.fpga-dev.com/altera-usb-blaster-with-ubuntu/}
\end{itemize}
}

USB Blaster diperlukan untuk mendownload program ke board.

Tidak perlu instalasi driver khusus untuk Linux.

Quartus Prime menggunakan driver \verb|usb_device|
pada Linux untuk mengakses USB-Blaster.

Secara default, yang dapat mengakses USB Blaster adalah root. Agar dapat digunakan
untuk user lain selain root, perlu setting tambahan untuk \verb|udev|.

Hubungkan USB-Blaster ke komputer.

Cek output dari \verb|dmesg| ketika USB-Blaster terhubung ke komputer.
{\small\setstretch{1.0}
\begin{minted}{text}
dmesg | tail
[29293.151196] usb 3-2: new full-speed USB device number 2 using xhci_hcd
[29293.282045] usb 3-2: New USB device found, idVendor=09fb, idProduct=6001
[29293.282056] usb 3-2: New USB device strings: Mfr=1, Product=2, SerialNumber=3
[29293.282061] usb 3-2: Product: USB-Blaster
[29293.282066] usb 3-2: Manufacturer: Altera
[29293.282070] usb 3-2: SerialNumber: 00000000
\end{minted}
}

Cek juga output dari \verb|lsusb|:
{\small\setstretch{1.0}
\begin{minted}{text}
lsusb
Bus 003 Device 002: ID 09fb:6001 Altera Blaster
\end{minted}
}

Berdasarkan output dari \verb|lsusb| diperoleh informasi
sebagai berikut:
{\small\setstretch{1.0}
\begin{minted}{text}
Vendor  ID: 09fb
Product ID: 6001
\end{minted}
}

Buat file baru: {\tt\small /etc/udev/rules.d/51-altera-usb-blaster.rules}
dengan konten sebagai berikut. (perlu akses root)
{\small\setstretch{1.0}
\begin{minted}{text}
SUBSYSTEM=="usb", ATTR{idVendor}=="09fb", ATTR{idProduct}=="6001", MODE="0666"
\end{minted}
}

Kemudian reload {\tt\small udev} (perlu akses root). Lepaskan koneksi USB
Blaster dari komputer sebelum melakukan perintah ini.
{\small
\begin{minted}{text}
udevadm control --reload
\end{minted}
}

Jika tidak ada error, maka seharusnya USB-Blaster dan board yang digunakan sudah
dapat diakses melalui komputer.
Tes apakah sudah dapat FPGA sudah dapat diakses
{\small
\begin{minted}{text}
<quartus_dir>/16.1/quartus/bin/jtagconfig
\end{minted}
}

Sebaiknya langkah berikut ini juga dilakukan agar daemon \verb|jtagd|
dapat mengenali
nama FPGA yang digunakan.
{\small
\begin{minted}{text}
cp <quartus_dir>/16.1/quartus/linux64/pgm_parts.txt /etc/jtagd/jtagd.pgm_parts
\end{minted}
}

Tes apakah FPGA sudah dapat diakses
{\small
\begin{minted}{text}
<quartus_dir>/16.1/quartus/bin/jtagconfig
\end{minted}
}


%{\color{red} XXX: Butuh konfirmasi lebih lanjut mengenai ini}

Perintah {\tt\small jtagconfig} akan secara otomatis menjalankan daemon
{\tt\small jtagd}.
Daemon ini dapat konflik dengan daemon {\tt\small jtagd} yang akan dijalankan
ketika menjalankan Quartus Prime Lite. Untuk amannya, pastikan bahwa
tidak ada daemon {\tt\small jtagd} yang berjalan sebelum menjalankan
Quartus Prime Lite.
